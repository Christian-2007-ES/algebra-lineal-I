% 02-sistemas.tex
\chapter{Sistemas de ecuaciones lineales y matrices}

\section{Sistemas de ecuaciones lineales}
	\begin{definicion}
		\begin{itemize}
			\item Una ecuación lineal de $n$ incógnitas $x_1,...,x_n$ en un cuerpo $K$ es una ecuación que puede escribirse en la forma $a_1x_1+\cdots+a_nx_n=b$ donde los coeficientes $a_1,...,a_n$ y el termino independiente (o constante) $b$ son elementos de $K$.
			\item Una solución de una ecuación lineal $a_1x_1+\cdots+a_nx_n=b$ es una $n$-tupla $(s_1,...,s_n)$ cuyos componentes satisfacen la ecuación cuando se sustituye $x_1=s_1,...,x_n=s_n$.
			\item Un sistema de ecuaciones lineales es un conjunto finito de ecuaciones lineales, cada una
			con las mismas variables. Una solución de un sistema de ecuaciones lineales es un vector que
			simultáneamente es una solución de cada ecuación en el sistema.
			\item El conjunto solución de un
			sistema de ecuaciones lineales es el conjunto de todas las soluciones del sistema. Al proceso
			de encontrar el conjunto solución de un sistema de ecuaciones lineales se le conocerá
			como “resolver el sistema”.
			\item Un sistema de ecuaciones lineales se llama compatible (o consistente) si tiene al menos una solución. Un sistema sin soluciones se llama incompatible (o inconsistente).
		\end{itemize}
	\end{definicion}
	\begin{definicion}
		Las operaciones elementales para un sistema de ecuaciones lineales son:
		\begin{itemize}
			\item Multiplicar una ecuación por un escalar no nulo.
			\item Intercambiar dos ecuaciones.
			\item Sumar a una ecuación otra multiplicada por un escalar.
		\end{itemize}
	\end{definicion}
	\begin{definicion}
		Un sistema de ecuaciones lineales se denomina homogéneo si el término constante en cada ecuación es cero.
	\end{definicion}
\section{Matrices}
	\begin{definicion}
		\begin{itemize}
			\item Una matriz en un cuerpo $K$ es un arreglo rectangular de elementos de $K$ llamados entradas, o elementos, de la matriz.
			\item El tamaño de una matriz es una descripción de los números de filas y columnas que tiene. Una matriz se llama $m\times n$ (dígase “$m$ por $n$”) si tiene $m$ filas y $n$ columnas.
			\item Una matriz de $1\times m$ se llama matriz fila (o vector fila), y una matriz de $n\times 1$ se llama matriz columna (o vector columna).
			\item La entrada de una matriz $A$ en la fila $i$ y la columna $j$ se denota mediante $a_{ij}$. De manera compacta una matriz $A$ se puede denotar mediante $(a_{ij})$.
			\item El conjunto de todas las matrices de tamaño $m\times n$ en un cuerpo $K$ se denota como $M_{mn}(K)$.
		\end{itemize}
	\end{definicion}
	\begin{notacion}
		Una matriz $A=(a_{ij})\in M_{mn}(K)$ tiene la forma
		\begin{center}
			$A=\begin{pmatrix}
				a_{11} & a_{12} & \cdots & a_{1n}\\
				a_{21} & a_{22} & \cdots & a_{2n}\\
				\vdots & \vdots & \ddots & \vdots\\
				a_{m1} & a_{m2} & \cdots & a_{mn}
			\end{pmatrix}$.
		\end{center}
		Si las columnas de $A$ son los vectores $a_1,...,a_n\in K^m$, entonces $A$ se puede representar como $A=\begin{pmatrix}
			a_1 & a_2 & \cdots & a_n
		\end{pmatrix}$. Si las filas de $A$ son $A_1,...,A_m\in K^n$, entonces $A$ se puede representar como
		\begin{center}
			$A=\begin{pmatrix}
				A_1\\
				A_2\\
				\vdots\\
				A_m
			\end{pmatrix}$.
		\end{center} 
	\end{notacion}
	\begin{definicion}
		\begin{itemize}
			\item Las entradas diagonales de $A=(a_{ij})$ son $a_{11},a_{22},a_{33},...,$ y si $m=n$ (esto es, si $A$ tiene el mismo número de filas que de columnas), entonces $A$ es una matriz cuadrada. Una matriz
			cuadrada cuyas entradas no diagonales sean todas cero es una matriz diagonal. Una
			matriz diagonal cuyas entradas diagonales sean todas iguales es una matriz escalar. Si el
			escalar en la diagonal es 1, la matriz escalar es una matriz identidad.
			\item El conjunto de todas las matrices cuadradas de tamaño $n\times n$ (o de orden $n$) en un cuerpo $K$ se denota como $M_{n}(K)$.
			\item Dos matrices son iguales si tienen el mismo tamaño y si sus entradas correspondientes son iguales.
		\end{itemize}
	\end{definicion}
	\begin{definicion}
		Para un sistema de $m$ ecuaciones y $n$ incógnitas en un cuerpo $K$
		\begin{center}
			$\left\{\begin{array}{c}
				a_{11}x_1+\cdots+a_{1n}x_n=b_1\\
				\cdots\\
				a_{m1}x_1+\cdots+a_{mn}x_n=b_m
			\end{array}\right.$
		\end{center}
		decimos que la matriz 
		\begin{center}
			$A=\begin{pmatrix}
				a_{11} & \cdots & a_{1n}\\
				\vdots & \ddots & \vdots\\
				a_{m1} & \cdots & a_{mn}
			\end{pmatrix}$.
		\end{center}
		es la matriz de coeficientes del sistema, y que la matriz 
		\begin{center}
			$(A|b)=\left(\begin{array}{ccc|c}
				a_{11} & \cdots & a_{1n} & b_1\\
				\vdots & \ddots & \vdots & \vdots\\
				a_{m1} & \cdots & a_{mn} & b_m
			\end{array}\right)$
		\end{center}
		es la matriz ampliada del sistema.
	\end{definicion}
	\begin{definicion}
		Una matriz está en forma escalonada por filas si satisface las
		siguientes propiedades:
		\begin{enumerate}
			\item Cualquier fila formada enteramente de ceros está en la parte de abajo. 
			\item En cada fila distinta de cero, la primera entrada no nula (llamado elemento pivote)
			está en una columna a la izquierda de cualquier elemento pivote debajo de ella.
		\end{enumerate}
	\end{definicion}
	\begin{definicion}
		Una matriz está en forma escalonada reducida por filas si satisface las siguientes propiedades:
		\begin{enumerate}
			\item Está en forma escalonada por filas.
			\item El elemento pivote en cada fila no nula es 1 (llamado 1 pivote).
			\item Cada columna que contiene un 1 pivote tiene ceros en todos los otros lugares.
		\end{enumerate}
	\end{definicion}
\section{Subespacios asociados a una matriz}
\section{Rango de una matriz}
	\begin{definicion}
		El rango de filas de una matriz $A$ es la dimensión del subespacio que generan sus filas, y el rango de columnas de una matriz $A$
	\end{definicion}
	Nota que el rango de filas de una matriz coincide con el número de filas no nulas de su forma escalonada reducida.
	\begin{teorema}
		El rango de filas de una matriz coincide con su rango de columnas.
	\end{teorema}
	\begin{proof}
		Sea $A$ una matriz cualquiera, y sea $R$ su forma escalonada reducida, tenemos que ver que $\dim(\Fil A)=\dim(\Col A)$. Como las operaciones elementales de filas no cambian el subespacio que generan estas, tenemos que $\Fil A=\Fil R$, llamemos $r=\dim(\Fil A)$, como $\dim(\Fil A)=\dim(\Fil R)=r$ tenemos que $R$ tiene $r$ filas no nulas, luego tiene $r$ 1-pivotes. Las columnas de $A$ que tiene 1-pivotes en su forma escalonada reducida son linealmente independientes, pues los sistemas $Ax=\bar0$ y $Rx=\bar0$ tienen las mismas soluciones, luego como las columnas de $R$ que no tienen pivote se pueden despejar como combinación lineal de las que tienen pivote tenemos que las columnas correspondientes a estas en $A$ se pueden despejar en las correspondientes a las columnas que tiene pivote, así pues, $\dim(\Col A)=\dim(\Col R)$. Finalmente, tenemos que
		\begin{center}
			$\dim(\Fil A)=\dim(\Fil R)=r=\dim(\Col R)=\dim(\Col A)$.
		\end{center}
	\end{proof}
	\begin{definicion}
		El rango de una matriz $A$ es la dimensión del subespacio que generan sus columnas, y se denota como $\rango(A)$.
	\end{definicion}
	\begin{proposicion}
		Sea $A$ una matriz, $\rango(A^t)=\rango(A)$.
	\end{proposicion}
	\begin{proof}
		Se tiene que $\rango(A^t)=\dim(\Col A^t)=\dim(\Fil A)=\rango(A)$.
	\end{proof}
\section{Operaciones con matrices}
\section{Matriz inversa}
\section{Factorización LU}