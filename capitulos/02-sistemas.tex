% 02-sistemas.tex
\chapter{Sistemas de ecuaciones lineales y matrices}

\section{Sistemas de ecuaciones lineales}

	\begin{definicion}
		
	\end{definicion}

\section{Matrices}
\section{Forma escalonada reducida de una matriz}
\section{Subespacios asociados a una matriz}
\section{Rango de una matriz}
	\begin{definicion}
		El rango de filas de una matriz $A$ es la dimensión del subespacio que generan sus filas, y el rango de columnas de una matriz $A$
	\end{definicion}
	Nota que el rango de filas de una matriz coincide con el número de filas no nulas de su forma escalonada reducida.
	\begin{teorema}
		El rango de filas de una matriz coincide con su rango de columnas.
	\end{teorema}
	\begin{proof}
		Sea $A$ una matriz cualquiera, y sea $R$ su forma escalonada reducida, tenemos que ver que $\dim(\Fil A)=\dim(\Col A)$. Como las operaciones elementales de filas no cambian el subespacio que generan estas, tenemos que $\Fil A=\Fil R$, llamemos $r=\dim(\Fil A)$, como $\dim(\Fil A)=\dim(\Fil R)=r$ tenemos que $R$ tiene $r$ filas no nulas, luego tiene $r$ 1-pivotes. Las columnas de $A$ que tiene 1-pivotes en su forma escalonada reducida son linealmente independientes, pues los sistemas $Ax=\bar0$ y $Rx=\bar0$ tienen las mismas soluciones, luego como las columnas de $R$ que no tienen pivote se pueden despejar como combinación lineal de las que tienen pivote tenemos que las columnas correspondientes a estas en $A$ se pueden despejar en las correspondientes a las columnas que tiene pivote, así pues, $\dim(\Col A)=\dim(\Col R)$. Finalmente, tenemos que
		\begin{center}
			$\dim(\Fil A)=\dim(\Fil R)=r=\dim(\Col R)=\dim(\Col A)$.
		\end{center}
	\end{proof}
	\begin{definicion}
		El rango de una matriz $A$ es la dimensión del subespacio que generan sus columnas, y se denota como $\rango(A)$.
	\end{definicion}
	\begin{proposicion}
		Sea $A$ una matriz, $\rango(A^t)=\rango(A)$.
	\end{proposicion}
	\begin{proof}
		Se tiene que $\rango(A^t)=\dim(\Col A^t)=\dim(\Fil A)=\rango(A)$.
	\end{proof}
\section{Operaciones con matrices}
\section{Matrices elementales}
\section{Matriz inversa}
\section{Factorización LU}