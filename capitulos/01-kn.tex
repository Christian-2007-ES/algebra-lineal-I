% 01-kn.tex
\chapter{Vectores de $K^n$}

\section{$K^n$ y sus operaciones}

	\begin{definicion}
		Sea $K$ un cuerpo y $n\in\N$. $K^n$ es el conjunto de todas las $n$-tuplas de longitud $n$ de elementos de $K$:
		\begin{center}
			$K^n=\{(x_1,...,x_n):x_i\in K,1\leq i\leq n\}$.
		\end{center}
		Para $(x_1,...,x_n)\in K^n$ y $i\in\{1,...,n\}$, decimos que $x_i$ es la $i$-ésima entrada de $(x_1,...,x_n)$.
	\end{definicion}
	A los elementos de $K$ los llamaremos escalares y a los de $K^n$ vectores $n$-tuplas. Normalmente escribiremos los elementos de $K^n$ en vertical, hecho del que veremos el porque más adelante.
	\begin{definicion}
		La suma de $(x_1,...,x_n),(y_1,...,y_n)\in K^n$ se define por
		\begin{center}
			$(x_1,...,x_n)+(y_1,...,y_n)=(x_1+y_1,...,x_n+y_n)$.
		\end{center}
	\end{definicion}
	\begin{definicion}
		El producto $(x_1,...,x_n)\in K^n$ y un escalar $t\in K$ se define por
		\begin{center}
			$t(x_1,...,x_n)=(tx_1,...,tx_n)$.
		\end{center}
	\end{definicion}
	\begin{notacion}
		Cuando estamos trabajando con $n$-tuplas $0$ denota a la $n$-tupla $(0,...,0)$.
	\end{notacion}
	\begin{notacion}
		Para simplificar $x\in K^n$ denotara a la $n$-tupla $(x_1,...,x_n)$.
	\end{notacion}
	La siguiente proposición muestra las  propiedades más elementales de los elementos de $K^n$.
	\begin{proposicion}
		Sean $u,v,w\in K^n$ y $t,s\in K$. Entonces:
		\begin{enumerate}
			\item $u+v=v+u$.
			\item $(u+v)+w=u+(v+w)$.
			\item $u+0=u$.
			\item $u+(-u)=0$.
			\item $t(u+v)=tu+tv$.
			\item $(t+s)u=tu+ts$.
			\item $(ts)u=t(su)$.
			\item $1u=u$.
		\end{enumerate}
	\end{proposicion}
	 Nótese que $(K^n,+)$ es un grupo abeliano con elemento neutro $0=(0,...,0)$. La siguiente proposición muestra algunas propiedades más, ya no tan triviales, de los elementos de $K^n$.
	 \begin{proposicion}
	 	Sean $u,v,w,v_1,...,v_m\in K^n$ y $t,t_1,...,t_m\in K^n$. Entonces:
	 	\begin{enumerate}
	 		\item $0u=0$.
	 		\item Si $u+v=u+w$, entonces $v=w$.
	 		\item $t0=0$.
	 		\item Si $tu=0$, entonces $t=0$ ó $u=0$.
	 		\item $(-t)u=-(tu)=-tu$, en particular, $(-1)u=-u$.
	 		\item En la expresión $t_1v_1+\cdots+t_mv_m$ no importa el orden en el que sumemos.
	 		
	 	\end{enumerate}
	 \end{proposicion}
\section{Subespacios de $K^n$}
	\begin{definicion}
		Sea $S\subseteq K^n$. Decimos que $S$ es un subespacio de $K^n$ si verifica que:
		\begin{enumerate}
			\item $S$ es no vacío.
			\item Para todos $u,v\in S$ se tiene que $u+v\in S$.
			\item Para todo $t\in K$ y $v\in S$ se tiene que $tv\in S$.
		\end{enumerate}
	\end{definicion}
	En tal caso escribimos $S\leq K^n$.
	\begin{observacion}
		La condición de que $S$ sea no vacío es equivalente a que $0\in S$.
	\end{observacion}
	\begin{observacion}
		Es inmediato que $\{0\}\leq K^n$ y $K^n\leq K^n$.
	\end{observacion}
	\begin{proposicion} Si para cada $i\in I$ consideremos un $S_i\leq K^n$, entonces $\displaystyle\bigcap_{i\in I}S_i\leq K^n$.
	\end{proposicion}
	\begin{definicion}
		Sean $S_1,...,S_m\leq K^n$, definimos la suma de $S_1,...,S_m$ como
		\begin{center}
			$S_1+\cdots+S_m=\{s_1+\cdots+s_m|s_i\in S_i,1\leq i\leq m\}$.
		\end{center}
	\end{definicion}
	\begin{proposicion}
		Si $S_1,...,S_m\leq K^n$, entonces $S_1+\cdots+S_m\leq K^n$.
	\end{proposicion}
\section{Combinaciones lineales y familias generadoras}
	\begin{definicion}
		Un vector $v\in K^n$ es combinación lineal de $v_1,...,v_m\in K^n$ si existen $t_1,...,t_m\in K$ tales que $v=t_1v_1+\cdots+t_mv_m$. Los escalares $t_1,...,t_m$ se llaman coeficientes de la combinación lineal.
	\end{definicion}
	\begin{definicion}
		Una lista o familia de vectores de longitud $m$ en $K^n$ es una $m$-tupla $(v_1,...,v_m)$ donde $v_1,...,v_m\in K^n$.
	\end{definicion}
	\begin{definicion}
		Sean $v_1,...,v_m\in K^n$, entonces el conjunto 
		\begin{center}
			$\gen(v_1,...,v_m)=\{t_1v_1+\cdots+t_mv_m|t_1,...,t_m\in K\}$
		\end{center}
		se llama generador de $v_1,...,v_m$. Si $S=\gen(v_1,...,v_m)$ se dice que la lista $(v_1,...,v_m)$ genera $S$, o que $(v_1,...,v_m)$ es una lista generadora de $S$.
	\end{definicion}
	\begin{proposicion}
		Si $v_1,...,v_m\in K^n$, entonces $\gen(v_1,...,v_m)\leq K^n$. Además, $\gen(v_1,...,v_m)$ es el subespacio más pequeño que contiene a $v_1,...,v_m$, es decir, si $S\leq K^n$ y $v_1,...,v_m\in S$, entonces $\gen(v_1,...,v_m)\subseteq S$.
	\end{proposicion}
	\begin{lema}
		Sean $v_1,...,v_m\in K^n$ y $u_1,...,u_k\in K^n$. Si $v_1,...,v_m\in\gen(u_1,...,u_k)$, entonces $\gen(v_1,...,v_m)\subseteq\gen(u_1,...,u_k)$. En particular, $\gen(v_1,...,v_m)=\gen(u_1,...,u_k)$ si y solo si $v_1,...,v_m\in\gen(u_1,...,u_k)$ y $u_1,...,u_k\in\gen(v_1,...,v_m)$.
	\end{lema}
	\begin{proof}
		Como $v_1,...,v_m\in\gen(u_1,...,u_k)$, dado $i\in\{1,...,m\}$ existen $t_{i1},...,t_{ik}\in K$ tales que $v_i=t_{i1}u_1+\cdots+t_{ik}u_k$. Sea $v\in\gen(v_1,...,v_m)$, entonces existen $c_1,...,c_m\in K$ tales que $v=c_1v_1+\cdots+c_mv_m$, luego 
		\begin{center}
			$v=\displaystyle\sum_{i=1}^{m}c_iv_i=\sum_{i=1}^{m}c_i(t_{i1}u_1+\cdots+t_{ik}u_k)=\sum_{i=1}^{m}c_i\sum_{j=1}^{k}t_{ij}u_j=\displaystyle\sum_{i=1}^{m}\sum_{j=1}^{k}c_it_{ij}u_j\in\gen(u_1,...,u_k)$.
		\end{center}
		Ahora supongamos que $v_1,...,v_m\in\gen(u_1,...,u_k)$ y $u_1,...,u_k\in\gen(v_1,...,v_m)$, por lo que acabamos de ver $\gen(v_1,...,v_m)\subseteq\gen(u_1,...,u_k)$ y $\gen(u_1,...,u_k)\subseteq\gen(v_1,...,v_m)$, luego $\gen(v_1,...,v_m)=\gen(u_1,...,u_k)$. Recíprocamente supongamos que $\gen(v_1,...,v_m)=\gen(u_1,...,u_k)$, dado $i\in\{1,...,m\}$ como $v_i=0v_1+\cdots+v_i+\cdots+0v_m\in\gen(v_1,...,v_m)=\gen(u_1,...,u_k)$ tenemos que $v_i\in\gen(u_1,...,u_k)$, por lo tanto, $v_1,...,v_m\in\gen(u_1,...,u_k)$. Análogamente $u_1,...,u_k\in\gen(v_1,...,v_m)$.
	\end{proof}
	\begin{teorema}
		$K^n=\gen(e_1,...,e_n)$ donde $e_i\in K^n$ es la $n$-tupla que tiene $0$ en todas sus entradas, salvo en la $i$, donde tienen un $1$.
	\end{teorema}
	\begin{proposicion}
		Si $v_1,...,v_m\in K^n$, entonces
		\begin{center}
			$\gen(v_1,...,v_i,...,v_j,...,v_m)=\gen(v_1,...,v_j,...,v_i,...,v_m)$
		\end{center} 
		para todos los $i,j\in\{1,...,m\}$ con $i\neq j$.
	\end{proposicion}
	\begin{proposicion}
		Si $v_1,...,v_m\in K^n$ y $t\in K,t\neq0$, entonces
		\begin{center}
			$\gen(v_1,...,v_i,...,v_m)=\gen(v_1,...,tv_i,...,v_m)$
		\end{center} 
		para todo $i\in\{1,...,m\}$.
	\end{proposicion}
	\begin{proposicion}
		Si $v_1,...,v_m\in K^n$ y $t\in K$, entonces
		\begin{center}
			$\gen(v_1,...,v_i,...,v_j,...,v_m)=\gen(v_1,...,v_i,...,v_j+tv_i,...,v_m)$
		\end{center} 
		para todos los $i,j\in\{1,...,m\}$.
	\end{proposicion}
	\begin{definicion}
		Un subespacio de $K^n$ se dice finitamente generado si existe una familia finita de vectores que lo genere.
	\end{definicion}
\section{Independencia lineal en $K^n$}
	\begin{definicion}
		Una lista $(v_1,...,v_m)$ de vectores de $K^n$ es linealmente dependiente (o ligada) si existen $t_1,...,t_m\in K$ con algún $t_i\neq0$, tales que
		\begin{center}
			$t_1v_1+\cdots+t_mv_m=0$.
		\end{center}
		Una lista $(v_1,...,v_m)$ de vectores de $K^n$ es linealmente independiente (o libre) si no es linealmente dependiente, es decir, si para todos los $t_1,...,t_m\in K$ tales que
		\begin{center}
			$t_1v_1+\cdots+t_mv_m=0$
		\end{center}
		necesariamente $t_1=\cdots=t_m=0$.
	\end{definicion}
	\begin{observacion}
		\begin{enumerate}
			\item La lista vacía es linealmente independiente por vacuidad.
			\item Toda lista que contiene a otra linealmente dependiente es linealmente dependiente.
			\item Toda lista contenida en otra linealmente independiente es linealmente independiente.
			\item En una lista linealmente independiente no puede estar el vector nulo, ni pueden haber vectores repetidos.
		\end{enumerate}
	\end{observacion}
	\begin{proposicion}
		La lista $(e_1,...,e_n)$ de vectores de $K^n$ es libre.
	\end{proposicion}
	\begin{proposicion}
		Una lista $(v_1,...,v_m)$ de vectores de $K^n$ es linealmente dependiente si y solo si existe $i\in\{1,...,m\}$ tal que $v_i$ es combinación lineal de los anteriores.
	\end{proposicion}
	\begin{observacion}
		Si una lista de vectores genera un espacio y está familia es linealmente dependiente, entonces hay un vector que es combinación lineal de los anteriores, si lo suprimimos la lista que resulta genera el mismo espacio.
	\end{observacion}
	\begin{lema}
		Una lista $v_1,...,v_m$ de vectores de $K^n$ es linealmente independiente si y solo si todo vector $v\in\gen(v_1,...,v_m)$ se puede escribir de manera única como combinación lineal de $v_1,...,v_m$.
	\end{lema}
	\begin{teorema}
		Toda familia $v_1,...,v_m$ de vectores en $K^n$ con $m>n$ es linealmente dependiente.
	\end{teorema}
\section{Bases y dimensión de subespacios de $K^n$}
	\begin{definicion}
		Una base de un subespacio $S$ de $K^n$ es una familia de vectores que genera $S$ y es linealmente independiente.
	\end{definicion}

	\begin{proposicion}
		Todas la bases de un subespacio finitamente generado de $K^n$ tienen la misma longitud.
	\end{proposicion}
	\begin{definicion}
		Sea $S\leq K^n$ finitamente generado, la longitud de todas las bases de $S$ se llama dimensión de $S$ y se denota por $\dim S$.
	\end{definicion}
		\begin{proposicion}
		Sea $S\leq K^n$, entonces $S$ es finitamente generado y $\dim S\leq n$. Además, $\dim S=n$ si y solo si $S=K^n$.
	\end{proposicion}
	\begin{proposicion}
		Sea $S\leq K^n$ con $\dim S=m$, y sea $\mathcal{F}$ una familia de vectores de $S$ de longitud $m$, entonces $\mathcal{F}$ es libre si y solo si es generadora. En particular, si $\mathcal{F}$ es libre o generadora, entonces es base de $S$.
	\end{proposicion}
