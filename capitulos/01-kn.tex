% 01-kn.tex
\chapter{Vectores de $K^n$}

\section{$K^n$ y sus operaciones}

	\begin{definicion}
		Sea $K$ un cuerpo, $t\in K$ y $n\in\N$, se definen
		\begin{center}
			$K^n=\{(x_1,...,x_n)|x_i\in K,1\leq i\leq n\}$,\\
			$(x_1,...,x_n)+(y_1,...,y_n)=(x_1+y_1,...,x_n+y_n)$,\\
			$t(x_1,...,x_n)=(tx_1,...,tx_n)$.
		\end{center}
	\end{definicion}
	A los elementos de $K$ los llamaremos escalares y a los de $K^n$ vectores $n$-tuplas.\\
	Normalmente escribiremos los elementos de $K^n$ en vertical, hecho que justificaremos más adelante.
	\begin{notacion}
		$\bar0=(0,...,0)$.
	\end{notacion}
	Entenderemos que si estamos trabajando con $n$-tuplas $\bar0$ es también una $n$-tupla. La siguiente proposición muestra las  propiedades más elementales de los elementos de $K^n$.
	\begin{proposicion}
		Sean $u,v,w\in K^n$ y $t,s\in K$. Entonces:
		\begin{itemize}
			\item $u+v=v+u$.
			\item $(u+v)+w=u+(v+w)$.
			\item $u+\bar0=u$.
			\item $u+(-u)=\bar0$.
			\item $t(u+v)=tu+tv$.
			\item $(t+s)u=tu+ts$.
			\item $(ts)u=t(su)$.
			\item $1u=u$.
		\end{itemize}
	\end{proposicion}
	 Nótese que $(K^n,+)$ es un grupo abeliano con elemento neutro $\bar0=(0,...,0)$. La siguiente proposición muestra algunas propiedades más, ya no tan triviales, de los elementos de $K^n$.
	 \begin{proposicion}
	 	Sean $u,v,w,v_1,...,v_m\in K^n$ y $t,t_1,...,t_m\in K^n$. Entonces:
	 	\begin{itemize}
	 		\item $0u=\bar0$.
	 		\item Si $u+v=u+w$, entonces $v=w$.
	 		\item $t\bar0=\bar0$.
	 		\item Si $tu=\bar0$, entonces $t=0$ ó $u=\bar0$.
	 		\item $(-t)u=-(tu)=-tu$, en particular, $(-1)u=-u$.
	 		\item En la expresión $t_1v_1+\cdots+t_mv_m$ no importa el orden en el que sumemos.
	 		
	 	\end{itemize}
	 \end{proposicion}
\section{Subespacios de $K^n$}
\section{Combinaciones lineales y familias generadoras}
\section{Independencia lineal en $K^n$}
\section{Bases y dimensión de subespacios de $K^n$}
